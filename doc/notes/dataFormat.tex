% ****** Note template ****** %

%\RequirePackage{lineno} 

\documentclass[aps,prd,superscriptaddress,floatfix]{revtex4}


\usepackage{graphicx}  % needed for 
\usepackage{dcolumn}   % needed for some tables
\usepackage{bm}        % for math
\usepackage{amssymb}   % for math

%\def\linenumberfont{\normalfont\small\sffamily}

\usepackage{psfrag}


\def\pp{$p\bar{p}$}


\topmargin=-1.1cm

\begin{document}

%\pacs{}


\title{  
\vspace{0.5cm}
Description of the Fermilab software school data objects
}

\author {Matthew Herndon}

\address{University of Wisconsin, Madison, Wisconsin, Fermi
  National Accelerator Laboratory, Illinois}

%\date{\today}

\begin{abstract}
\vskip 0.5cm
\noindent
This document describes the data objects used in the Fermilab software school.
The objects include a StripSet with raw digitized data for the strip sensors
of the toy simulated detector as well as Hits, Tracks, GenTracks and associated
HitSets, TracksSets and GenTrackSets.  In addition, the formats used to store the
data objects to file are described.
\end{abstract}
\maketitle

%\tableofcontents
%\setpagewiselinenumbers
%%%\modulolinenumbers[10]
%%%\linenumbers

\vspace{0.3cm}

\section{Introduction}
Each objects is described below including the object's class and the methods and
format used to write the objects.  Sets are described with the associated data
object.
\\

The detector geometry of the strip sensors is defined in ~\cite{detectorGeometry}.


\section{The StripSet}
\subsection{StripSet class}
Implemented in the StripSet.hh and StripSet.cc files in the DataObjects directories.
\\

The StripSet contains a std::vector of typedef LayerStripMap with one map per layer of the detector.
The detector is dynamically defined from a geometry file and the vector and size are initialized
at run time.  Each layerStripMap is a map of key int stripNumber and adc value.  This allows
the strip data to be stored in a sparse and ordered format.
\\

Access methods:
\begin{itemize}
\item int getNumberLayers() To assist in iterating over the vector or directly over the layerMaps
\item getStrips(): const reference access to the vector using a generic method. The underlying vector container could be 
changed to several different container types without effecting the user interface.
\item getLayerStripMap(int layer): const reference access to a the strip data from a single layer. The method explicitly 
indicates that a map is retrieved since maps have access methods methods that are not common to all containers.
\item insertStrip(unsigned int layer, int stripNumber, int acd):  To insert data into the set.
\item print(ostream\&): To print to an a user specified ostream.
\end{itemize}

\subsection{StripSet IO}
Implemented in the StripSetIO.hh and StripSetIO.cc files in the Algorithms directories.
\\

Input and output to disk in controlled by the StripSetIO helper class.  Reading and writing
are performed by the readEvent and writeEvent functions.
\\

The data is stored, or ``streamed'', to disk in bit packed format to illustrate a typical way date is stored in compact
format in many experiments.  The data is stored in individual bytes or pairs of bytes for information
that does not fit in one byte.  The data structure is as follows.
\begin{itemize}
\item Strips: In clear test to identify the Set
\item  1 byte, version: The version is checked on input to make sure the correct streamer code is being used.
\item - 1 byte, layer number: 0-9 (repeated once per layer)
\item - 1 byte, number of strips: max 256
\item -- 2 bytes. strip number and adc: strip number 11 bytes, max 2048, and acd 5 bytes, max 32. (repeated number of strips times)
\end{itemize}

Note layer numbers are not necessary to the
structure but are included to facilitate synchronization with write and read operation..
\\

A number of helper functions exist in the StripHitFunctions file in Geometry directories.  These functions
can convert from strip number and layer to local and global positions and back again and identify whether a strip number is valid
on a given layer.

\subsection{binary input and output}
Data input and output is controlled via standard C++ library binary input and
output functions. std::ofstream, ifstream.  Byte writing is performed for write and read functions as:

\begin{itemize}
\item stripdata.write reinterpret\_cast const char *(\&myInt), 1);
\item stripdata.read (myCharByte, 1);
\end{itemize}
\

\section{The GenHit and GenHitSet}
\subsection{GenHit class}
Implemented in the GenHit.hh and GenHit.cc files in the DataObjects directories.
\\

The GenHit class contains:\\

\begin{itemize}
\item TVector3 \_hitPosition
\item int \_layer
\item Int \_trackNumber
\end{itemize}


The class server for generator GenHits only.\\
\\  

Constructor:

\begin{itemize}
\item Hit(hitPosition,layer,trackNumber)
\end{itemize}

Access methods:
\begin{itemize}
\item getHitPosition():... names correspond to the data members
\item getLayer()
\item getTrackNumber()
\item print(ostream\&): To print to an a user specified ostream.
\end{itemize}

A number of helper functions exist in the StripHitFunctions file in Geometry directories.  These functions
can convert from hit position to local and strip number position or back identify whether a hit position at a sensor
plane is within the active area of the sensor.


\subsection{GenHitSet class}
Implemented in the GenHitSet.hh and GenHitSet.cc files in the DataObjects directories.
\\

The GenHitSet contains a typedef GenHitSetContainer of type std::vector of GenHits.
\\

Access methods:

\begin{itemize}
\item getGenHits(): const reference access to the vector using a generic method.
The vector index indexes the GenHit number to allow direct access to a hit of a known number.

\item insertGenHit(GenHit): Provided to insert data into the set.

\item print(ostream\&): To print to an a user specified ostream that iterates over
the GenHits calling the print method of each GenHit..
\end{itemize}

\subsection{GenHitSet IO}
Implemented in the GenHitSetIO.hh and GenHitSetIO.cc files in the Algorithms directories.
\\

Input and output to disk in controlled by the GenHitSetIO helper class.  Reading and writing
are performed by the readEvent and writeEvent functions.
\\

The data is stored in text format for simplicity.

\begin{itemize}
\item  Hits: to identify the set
\item version: The version is checked on input to make sure the correct streamer code is being used.
\item number of hits
\item hit number (repeated number of hits times)
\item - x position
\item - y position
\item - z position
\item - layer number
\item - track number 
\end{itemize}

\section{The Hit and HitSet}
\subsection{Hit class}
Implemented in the Hit.hh and Hit.cc files in the DataObjects directories.
\\

The Hit class contains:\\

\begin{itemize}
\item TVector3 \_hitPosition
\item int \_layer
\item int \_numberStrips
\item int \_charge
\item bool \_isGoodHit
\item double \_resolution 
\end{itemize}


The class server for reconstructed Hits only.   Associated information such as Hit resolutions
are read from DetectorGeometry though a resolution variable exists per hit to allow for any resolution model
to be developed.  Also note that a hit is classified of as ``bad'' if it has a large ADC
value or more than two associated strips indicating that it is actually due to several overlapping hits from
different tracks.
\\

Constructors:

\begin{itemize}
\item Hit(hitPosition,layer,numberStrips,charge,goodhit,resolution)
\end{itemize}

Access methods:
\begin{itemize}
\item getHitPosition():... names correspond to the data members
\item getLayer()
\item getNumberStrips()
\item getCharge()
\item bool isGoodHit()
\item getResolution
\item print(ostream\&): To print to an a user specified ostream.
\end{itemize}

A number of helper functions exist in the StripHitFunctions file in Geometry directories.  These functions
can convert from hit position to local and strip number position or back identify whether a hit position at a sensor
plane is within the active area of the sensor.
\\

\subsection{HitSet class}
Implemented in the HitSet.hh and HitSet.cc files in the DataObjects directories.
\\

The HitSet contains a typedef HitSetContainer of type std::vector of Hits.
\\

Access methods:

\begin{itemize}
\item getHits(): const reference access to the vector using a generic method.
The vector index indexes the Hit number to allow direct access to a hit of a known number.

\item insertHit(Hit): Provided to insert data into the set.

\item print(ostream\&): To print to an a user specified ostream that iterates over
the Hits calling the print method of each Hit..
\end{itemize}

\subsection{HitSet IO}
Implemented in the HitSetIO.hh and HitSetIO.cc files in the Algorithms directories.
\\

Input and output to disk in controlled by the HitSetIO helper class.  Reading and writing
are performed by the readEvent and writeEvent functions.
\\

The data is stored in text format for simplicity.

\begin{itemize}
\item  Hits: to identify the set
\item version: The version is checked on input to make sure the correct streamer code is being used.
\item number of hits
\item hit number (repeated number of hits times)
\item - x position
\item - y position
\item - z position
\item - layer number
\item - number of strips
\item - charge
\item - isGoodHit
\item - resolution
\end{itemize}

\section{The GenTrack and GenTrackSet}
\subsection{GenTrack class}
Implemented in the GenTrack.hh and GenTrack.cc files in the DataObjects directories.
\\

The GenTrack class contains:
\begin{itemize}
\item TLorentzVector \_lorentzVector
\item int \_charge
\item TVector3 \_dr, point of closest approach to the reference point, 0,0,0
\end{itemize}

Constructor:

\begin{itemize}
\item GenTrack(TLorentzVector,int charge, TVector3 dr position of closest approach)
\item{itemize}
\end{itemize}


Access:
\begin{itemize}
\item getLorentzVector():
\item getPosition(): \_dr

\item makeHelix(TVector3 bField,curvatureC): function which takes the magnetic field and the curvature constant and returns
a Helix which can be used for operations like finding intersections with layers.  Note the magnetic field
value is needed since it may not be uniform and the radius of curvature would be different at different points.
With the standard bField from the DetectorGeometry you get the radius of curvature at the origin.

\item print(ostream\&) to print to an a user specified ostream.
\end{itemize}


\subsection{GenTrackSet class}
Implemented in the GenTrackSet.hh and GenTrackSet.cc files in the DataObjects directories.
\\

The GenTrackSet contains a typedef GenTrackSetContainer of type std::vector of GenTracks.
\\
Access:

\begin{itemize}
\item getGenTracks(): const reference access to the vector using a generic method.
The vector index indexes the GenTrack number to allow direct access to a hit of a known number.

\item insertTrack(GenTrack): Provided to insert data into the set.

\item print(ostream\&): to print to an a user specified ostream that iterates over
the GenTracks calling the print method of each GenTrack..
\end{itemize}


\subsection{GenTrackSet IO}
Implemented in the GenTrackSetIO.hh and GenTrackSetIO.cc files in the Algorithms directories.
\\

Input and output to disk in controlled by the GenTrackSetIO helper class.  Reading and writing
are performed by the readEvent and writeEvent functions.
\\

The data is stored in text format for simplicity.

\begin{itemize}
\item  GenTracks: to identify the set
\item version: The version is checked on input to make sure the correct streamer code is being used.
\item number of GenTracks
\item - GenTrack number (repeated number of GenTrackNumber times)
\item - charge
\item - px
\item - py
\item - pz
\item - E
\item - x position
\item - y position
\item - z position
\end{itemize}

\section{The Track and TrackSet}
\subsection{Track class}
Implemented in the Track.hh and Track.cc files in the DataObjects directories.
\\

The Track class contains:
\begin{itemize}
\item Helix \_helix, 5 track parameter helix
\item int \_covMatrix, 5x5 parameter covariance matrix
\item double \_chi2
\item int \_nDof
\item \_trackHits, a typedef TrackHitContainer of type vector of ints with hit index numbers.
\end{itemize}


Constructor:
\begin{itemize}
\item Tracks are constructed by the full set of parameters above.  These parameters
are generated by the BuildTrack helper function in the Algorithms directories.
That class fits a track based the input hits.
\item setHelix() expert method for use in TrackFit where the helix must be updated frequently
\end{itemize}


Access:
\begin{itemize}
\item getHelix() get Helix object with track parameters
\begin{itemize}
\item Helix accessor functions called as getHelix().getDr() ...:
\item getDr(): Impact parameter, distance in the x-y place to the reference point (0,0,0) at the point of closest
approach. Signed positive to indicate the reference point is within the circle or negative is outside.
\item getPhi0(): Phi angle from the reference point to the point of closest approach.
\item getKappa(): inverse Pt signed to indicate handedness + right handed (negative charge) - left handed
(positive charge)
\item getDz(): z distance to reference point at point of closest approach
<<<<<<< HEAD
\item getTanL(): pZ/pT or inverse cot(theta) where theta is the dip angle which is zero for track perpendicular
to the z axis and 90 degrees for parallel.
\end{itemize}
The Helix helix parameter follow the convention of~\cite{helix} and are described in detail there.\\
Note that helix parameters are always referenced from (0,0,0) in the parametrization used in this code.
\item getLorentzVector() get a root TLorentzVector object based on the track parameters
=======
\item getTanL(): pZ/pT or cot(theta) where theta is the dip angle which is 90 for track perpendicular
to the z axis and 0 degrees for parallel to the position z axis.
\item getAlpha() get inverse curvature constant in the magnetic field at the origin
\item getRadiusOfCurvatureAtOrigin()
\item getRadiusOfCurvature(TVector3 bField) at an arbitrary point with field bField
\item getPT()
\item getPZ()
\item getCotTheta() same as tanL
\item getCosTheta() where theta is the dip angle
\item getSinTheta() where theta is the dip angle 
\end{itemize}
The Helix helix parameter follow the convention of~\cite{helix} and are described in detail there.\\
Note that helix parameters are always referenced from (0,0,0) in the parametrization used in this code.
\item getSigmDR() ... unceartianties for the five track parameters from the covariance matrix
\item getCharge()
\item getCovMatrix()
\item getChi2()
\item getNDof()
\item getChi2Prob()
\item getLorentzVector() get a root TLorentzVector object based on the track parameters
\item getHits() get TrackHitContainer vector of hit indices's
\item numberHits()
>>>>>>> master
\item print(ostream\&): to print to an a user specified ostream.

\end{itemize}


\subsection{TrackSet class}
Implemented in the TrackSet.hh and TrackSet.cc files in the DataObjects directories.
\\

The TrackSet is a typedef TrackSetContainer of type std::vector of Tracks.
\\

Access:
\begin{itemize}
\item getTracks(): const reference access to the vector using a generic method.
The vector index indexes the Track number to allow direct access to a hit of a known number.

\item insertTrack(Track): provided to insert data into the set.


\item print(ostream\&): to print to an a user specified ostream that iterates over
the Tracks calling the print method of each Track..
\end{itemize}


\subsection{TrackSet IO}
Implemented in the TrackSetIO.hh and TrackSetIO.cc files in the Algorithms directories.
\\

Input and output to disk in controlled by the TrackSetIO helper class.  Reading and writing
are performed by the readEvent and writeEvent functions.
\\

The data is stored in text format for simplicity.

\begin{itemize}
\item Tracks: to identify the set
\item version: The version is checked on input to make sure the correct streamer code is being used.
\item number of GenTracks
\item - Track number (repeated number of GenTrackNumber times)
\item - charge
\item - px
\item - py
\item - pz
\item - E
\item - x position
\item - y position
\item - z position
\item - number of Hits
\item -- Hit index numbers (repeated number of Hits times)
\end{itemize}

The hit list could be used alone to restore the track using BuildTrack.  Currently TrackSetIO is not used.

\begin{thebibliography}{99}

\bibitem{detectorGeometry}
detectorGeometry.pdf note, M. Herndon (2014)

\bibitem{helix} 
  B.~Li, K.~Fujii and Y.~Gao,
  %``Kalman-filter-based track fitting in non-uniform magnetic field with segment-wise helical track model,''
  Comput.\ Phys.\ Commun.\  {\bf 185}, 754 (2014)
  [arXiv:1305.7300 [physics.ins-det]].
  %%CITATION = ARXIV:1305.7300;%%
  %1 citations counted in INSPIRE as of 02 Aug 2014


\end{thebibliography}
% 
\end{document}